%!TEX root = ../template.tex
%%%%%%%%%%%%%%%%%%%%%%%%%%%%%%%%%%%%%%%%%%%%%%%%%%%%%%%%%%%%%%%%%%%%
%% abstract-en.tex
%% NOVA thesis document file
%%
%% Abstract in English
%%%%%%%%%%%%%%%%%%%%%%%%%%%%%%%%%%%%%%%%%%%%%%%%%%%%%%%%%%%%%%%%%%%%

\typeout{NT FILE abstract-en.tex}%

Regardless of the language in which the dissertation is written, usually there are at least two abstracts: one abstract in the same language as the main text, and another abstract in some other language.

The abstracts' order varies with the school.  If your school has specific regulations concerning the abstracts' order, the \gls{novathesis} (\LaTeX) template will respect them.  Otherwise, the default rule in the \gls{novathesis} template is to have in first place the abstract in \emph{the same language as main text}, and then the abstract in \emph{the other language}. For example, if the dissertation is written in Portuguese, the abstracts' order will be first Portuguese and then English, followed by the main text in Portuguese. If the dissertation is written in English, the abstracts' order will be first English and then Portuguese, followed by the main text in English.
%
However, this order can be customized by adding one of the following to the file \verb+5_packages.tex+.

\begin{verbatim}
    \ntsetup{abstractorder={<LANG_1>,...,<LANG_N>}}
    \ntsetup{abstractorder={<MAIN_LANG>={<LANG_1>,...,<LANG_N>}}}
\end{verbatim}

For example, for a main document written in German with abstracts written in German, English and Italian (by this order) use:
\begin{verbatim}
    \ntsetup{abstractorder={de={de,en,it}}}
\end{verbatim}

Concerning its contents, the abstracts should not exceed one page and may answer the following questions (it is essential to adapt to the usual practices of your scientific area):

\begin{enumerate}
  \item What is the problem?
  \item Why is this problem interesting/challenging?
  \item What is the proposed approach/solution/contribution?
  \item What results (implications/consequences) from the solution?
\end{enumerate}

% Palavras-chave do resumo em Inglês
% \begin{keywords}
% Keyword 1, Keyword 2, Keyword 3, Keyword 4, Keyword 5, Keyword 6, Keyword 7, Keyword 8, Keyword 9
% \end{keywords}
\keywords{
  One keyword \and
  Another keyword \and
  Yet another keyword \and
  One keyword more \and
  The last keyword
}
