%!TEX root = ../template.tex
%%%%%%%%%%%%%%%%%%%%%%%%%%%%%%%%%%%%%%%%%%%%%%%%%%%%%%%%%%%%%%%%%%%
%% chapter1.tex
%% NOVA thesis document file
%%
%% Chapter with introduction
%%%%%%%%%%%%%%%%%%%%%%%%%%%%%%%%%%%%%%%%%%%%%%%%%%%%%%%%%%%%%%%%%%%

\typeout{NT FILE 04_workplan.tex}%

\chapter{Work Plan}\label{cha:work_plan}

This thesis investigates how geometric inductive biases (GIBs) can be used to
build interpretable, efficient, and reliable machine learning systems for 3D
scene understanding and power grid inspection. The central hypothesis is that
embedding geometric structure into learning models not only improves
transparency and performance but also supports the development of decision
support tools suited for high-stakes environments.
%
The contributions span three main lines of work. The first is the development
of TS40K, a large-scale benchmark tailored to rural power grid inspection. The
second includes SCENE-Net and SCENE-Net v2, which explore the integration of
GIBs into white-box and grey-box semantic segmentation models. The third
focuses on cost-aware machine learning for inspection automation, proposing how
risk-sensitive pipelines can be deployed with human-in-the-loop validation.
%
The final stage of the thesis is the development of GIBLi, a modular and
scalable GIB-based feature extractor for raw point clouds. GIBLi aims to align
geometric reasoning with modern point-based architectures and will be tested
across several 3D benchmarks to assess its generalization and effectiveness.

Table~\ref{tab:workplan} outlines the remaining work scheduled through the
dissertation submission in October 2025. The immediate focus is on finalizing
the GIBLi-layer and integrating it into point-based networks. This will be
followed by benchmarking, paper submission, and the writing of the final
dissertation.

\begin{table}[h]
    \centering
    \begin{tabular}{lp{0.65\textwidth}}
        \hline
        \textbf{Time Interval} & \textbf{Work}                                                                          \\ \hline
        May – June 2025        & Finalize design and implementation of the GIBLi-layer.             \\ \hline
        June – July 2025       & Integrate GIBLi into SOTA models and evaluate it on benchmarks. \\ \hline
        July – August 2025     & Prepare GIBLi manuscript for submission to a top-tier venue.   \\ \hline
        August – October 2025  & Write and revise the PhD dissertation. \\ \hline
    \end{tabular}
    \caption{Tentative plan for the remaining work.}
    \label{tab:workplan}
\end{table}