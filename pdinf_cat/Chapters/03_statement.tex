%!TEX root = ../template.tex
%%%%%%%%%%%%%%%%%%%%%%%%%%%%%%%%%%%%%%%%%%%%%%%%%%%%%%%%%%%%%%%%%%%
%% chapter1.tex
%% NOVA thesis document file
%%
%% Chapter with introduction
%%%%%%%%%%%%%%%%%%%%%%%%%%%%%%%%%%%%%%%%%%%%%%%%%%%%%%%%%%%%%%%%%%%

\typeout{NT FILE 03_statement.tex}%

\chapter{Research Statement}\label{cha:statement}



% ------------------------------------
\section{Geometric Inductive Biases for 3D Scene Understanding}

\subsection{Problem Statement}
% → Define the core problems: local feature ambiguity, rotational variance, data inefficiency.
% → Ground the issues in literature (connect to previous chapter) without repeating the full survey.

\subsection{Research Objectives}
% → List high-level goals your work aims to achieve (robustness, interpretability, efficiency).
% → Formulate as specific questions or targets to address the problems above.

\subsection{Proposed Approach}
% → Outline how geometric inductive biases will be integrated into learning pipelines.
% → Mention the role of local geometric features, neighborhood structures, and equivariant designs.






% ------------------------------------
\section{Stochastic Augmented Lagrangian for Dynamic Regularization}

\subsection{Purpose and Context}
% → Explain why a regularization method is needed (e.g., training stability, geometry constraints).
% → State where this method fits in your pipeline (e.g., loss formulation or constraint enforcement).

\subsection{Research Objectives}
% 

\subsection{Proposed Approach}
% → Describe the planned mechanism: stochastic constraints, dynamic regularization, Lagrangian updates.
% → Keep it high-level but clearly connected to your goals (e.g., better generalization, robustness).



% ------------------------------------
\section{Application Domain: Power Grid Inspection}

\subsection{Research Context}
% → Briefly describe the importance of power grid monitoring and current inspection limitations.
% → Emphasize the lack of public datasets and challenges in deploying 3D vision in this setting.

\subsection{Relevance to Proposed Work}
% → Link your methods to this domain: sparse/noisy data, safety-critical nature, generalization under constraint.
% → Describe how your research could be tested or validated in this scenario.

\subsection{Proposed Approach}


\subsection{Expected Impact}
% → Specify the potential practical outcomes (e.g., reduced manual inspection, anomaly detection).
% → Mention any planned datasets, evaluation pipelines, or industry collaborations.



% ------------------------------------
\section{Summary and Outlook}
% → Recap the key problems, goals, and proposed strategies.
% → Briefly prepare the reader for the next chapter (e.g., Work Plan).